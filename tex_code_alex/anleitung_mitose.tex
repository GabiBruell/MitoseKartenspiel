\documentclass[fleqn,a4paper,12pt,titlepage,headsepline]{scrartcl}

\usepackage{gauss} 


\usepackage[footskip=30pt]{geometry}

\usepackage{bm}
%für dicke Buchstaben im Mathemodus

\usepackage{bigints}

\usepackage{ziffer}

\usepackage[T1]{fontenc}
%das ist für europäische Zeichen

\usepackage[utf8]{inputenc}
%das ist für Sonderzeichen

\usepackage{rotating}
\usepackage{multirow}
\usepackage{tabularx}
\usepackage{longtable}

\usepackage{booktabs}

\usepackage{lmodern}
%das ist für Umlaute

\usepackage[ngerman]{babel}
%das ist für deutsche Sprache


\geometry{
	left=2.5cm,
	right=2.5cm,
	top=2cm,
	bottom=2cm,
	bindingoffset=5mm
}


\usepackage{graphicx}
%für Bilder


\usepackage{graphicx}
\usepackage{float}

\usepackage[center]{caption}

\usepackage{wrapfig}



\linespread{1.25}



\begin{document}
	\shorthandoff{"}
\begin{huge}
Anleitung
\end{huge}\\
	\textbf{Spielvorbereitung}\\
Die Karten werden gut gemischt und jeder Spieler erhält 5 Karten.
Die übrigen Karten werden als verdeckter Stapel in die
Tischmitte gelegt.\\
\textbf{Spielverlauf}\\
Der Spieler, der die Beschreibung der Prophase hat (blaue Karte), beginnt und legt
diese Karte offen in der Tischmitte aus. Anschließend
verläuft das Spiel im Uhrzeigersinn weiter. Hat kein
Spieler die Beschreibung der Prophase auf der Hand, beginnt derjenige,
der die Prophase in einer anderen Farbe besitzt. Als Rangfolge gilt: Zeichnung-Fachbegriff-Mikroskopbild. Hat kein Spieler eine Prophase
auf der Hand, müssen die Karten neu gemischt und
ausgeteilt werden.
Der nächste Spieler ist nun an der Reihe und hat 3
Möglichkeiten:\\
1. Der Spieler legt an die bereits ausgelegte Prophase
links oder rechts eine gleichfarbige Phase, die in den Mitosezyklus vor oder nach der Prophase gehört.\\ 
2. Der Spieler legt über oder unter der ausgespielten
Prophase eine weitere Prophase an und eröffnet so eine
weitere Reihe.\\
3. Der Spieler kann nicht anlegen und muss daher
eine Karte vom verdeckten Stapel ziehen. Passt
diese Karte, darf er sie sofort anlegen und der
nächste Spieler ist an der Reihe. Passt die Karte
nicht, muss er eine weitere Karte ziehen, maximal
jedoch 3 Stück.\\
\textbf{Weiterhin gelten folgende Regeln:}\\
• Jeder Spieler kann nicht nur eine, sondern beliebig
viele passende Karten auslegen, wenn er an der
Reihe ist.\\
• Es müssen nicht alle passenden Karten ausgelegt
werden. Ein beliebter Trick ist es, bestimmte Karten
so lange wie möglich zurückzuhalten, um dadurch
andere Spieler zu blockieren. Der Spieler an der Reihe muss jedoch mindestens eine Karte ausspielen,
wenn er kann.\\
• Ist der Nachziehstapel aufgebraucht, kann keine
Karte mehr nachgezogen werden. In diesem Fall
muss der Spieler in dieser Runde einfach passen.
Das Spiel endet, sobald der erste Spieler
alle seine Karten anlegen konnte. Dieser
Spieler hat gewonnen.\\
\textbf{Spielvariation}\\
Das Spiel kann auch mit einem doppelten Kartenset gespielt werden. Die vier möglichen Stapel bleiben bestehen. Pro Stapel existiert dann jede Karte zweimal, sodass es pro Stapel am Ende zwei Mitosezyklen gibt.
	
	\end{document}